\chapter{U--\protect\NoCaseChange{Mo} fuel: a brief introduction}

Since the discovery of fission in 1938, the world has seen the impact of nuclear power on the modern world. The primary fuel for fission is $^{235}$U. Naturally occurring uranium consists of a combination of different isotopes; about 99.3\% $^{238}$U,  0.7\%$^{235}$U and trace amounts of $^{234}$U. Uranium can be `enriched' in the $^{235}$U isotope by using various complicated processes that mainly uses the defference in masses and/or other physical properties. Uranium that has a concentration of $^{235}$U from natural level to 20\% is called low enriched uranium (LEU). Typical power reactor fuel has $^{235}$U assays below 5\%.

After the Manhattan project, the enriched uranium ($>$90\%$^{235}$U) has been used for many peaceful usage in scientific research and radioisotope production. Nuclear fuels are characterized by a unique combination of physics, engineering, safety requirements, social and environmental factors, and international concerns. The purpose of nuclear safety is to prohivit the uncontrolled release of the radioactive isotopes to the environment and ensuring safe use of the fissile material. Highly enriched uranium (HEU) fuel is also a proliferation concern. All enrichments above 20\% $^{235}$U are considered HEU~\cite{international2005iaea}

%\section{Introduction}\label{sec_ch1_intro}
Proliferation concerns about HEU resulted a global effort, led by USA, to eradicate the civil uses of HEU in research and test reactors. One of the important programmes is known as RERTR.
The Reduced Enrichment for Research and Test Reactors (RERTR) program was initiated in the USA in the late 1970s to develop new nuclear fission fuels to replace high-enriched uranium (HEU)~\cite{travelli1980current,snelgrove1997development}\@.  The RERTR program is now managed by the U.S. National Nuclear Security Administration (NNSA) office of Material Management and Minimization (M$^3$). The development of low-enrichment uranium (LEU) fuels for high-performance reactors is an important nonproliferation initiative~\cite{m3web}.
%``\textit{Material Management and Minimization program reduces the risk of highly enriched uranium and plutonium falling into the hands on nonstate actors by minimizing the use of and, when possible, eliminating weapons-usable nuclear material around the world"}. 
This initiative has completed a total of 69 reactors conversions to the use of LEU fuel, and 26 reactor facilities have been verified to have been shut down~\cite{wilson2017us}. The conversion of six domestic high performance research reactors~\footnote{Advance Test Reactor at INT, Idaho; Advanced Test Reactor Assembly at INL, Idaho; High Flux Isotope Reactor at ORNL, Tennessee; Massachusetts Institute of Technology Reactor, Massachusetts; National Bureau of Standards Reactor in Gaithersburg, Maryland; University of Missouri Research Reactor in Columbia, Missouri.} that still use highly enriched uranium fuel is yet to be achieved. Due to their unique operating conditions, converting these six reactors is not easy and created a plethora of nuclear engineering challenges. These conversion process may take longer time periods, but developing a new LEU fuel is essential to ensure better performance and limiting nuclear proliferation. The current timeline to for the conversion is estimated to be 10--16 years~\cite{national2016reducing, national2012progress}. 



Research reactors operate at relatively low peak fuel temperatures, but are required to meet fuel performance requirements at high burnup. A typical peak fuel centerline temperature is around 250\textdegree C~\cite{meyer2014irradiation}. For a research reactor fission densities are usually in the range of $3\times10^{21}$ to $6\times10^{21}$ fissions/cm$^3$. In some cases, peak fuel fission density exceeds $7\times10^{21}$ fissions/cm$^3$, requiring a higher number of initial $^{235}$U atoms. One of the main requirements of LEU fuels is increased uranium density, such as that found in metallic uranium, to offset the decrease in \textsuperscript{235}U enrichment. There are a fewer number of uranium alloys that have the combination of high uranium density and stable fuel behavior to the high burnup to replace the high power density reactors.  Metallic uranium is thought to have sufficient density, but the orthorhombic crystal structure of \textalpha-U
and the anisotropic fuel swelling that results make it unattractive as a fuel~\cite{pugh1961swelling}.
Uranium alloys that retain the high-temperature \textgamma-phase, which is body-centered cubic, are more suitable for reactor fuel due to their more isotropic radiation-induced swelling behavior compared with  \textalpha-uranium~\cite{kittel1993history}.

Various uranium alloys have been tested as alternative metallic fuels under reactor operating conditions, including U$_6$Fe and U$_6$Mn~\cite{meyer2000irradiation,hofman1987irradiation}.
%The U-Mo alloy has been identified as a high-performance fuel due to its high uranium density and low neutron capture cross-section~\cite{ewh2010microstructural,smirnova2013ternary,rest2009analysis,landa2013density}.
Elements such as molybdenum (Mo), niobium (Nb), titanium (Ti), and zirconium (Zr) have also been tried as alloying elements because of their solubility in \textgamma-uranium~\cite{donze1959stabilisation,giraud1973formation,lopes2013mechanical}. Molybdenum stabilizes uranium's \textgamma-phase at concentrations near the eutectoid point, lowering the phase transition temperature from 776~\textdegree C for pure uranium (corresponding to the \textbeta--\textgamma\ allotropic point) to the eutectoid point of 555~\textdegree C for 11.1~percent molybdenum in \textgamma-uranium~\cite{ASM-Alloy-Mo,Berche2011}. To take advantage of this, uranium alloyed with 10 wt$\%$ molybdenum (U-10Mo) is currently being developed as a potential high-density LEU fuel for high-performance research reactors. 

Before the current interest in U--Mo metallic fuel, some of the earlier nuclear reactors used metallic fuel because of the combination of high uranium density and metallic properties. The Godiva IV pulsed reactors at Los Alamos (initially known as \textit{Lady Godiva}) used U--Mo alloys, which date back to 1960. The Fast Burst Reactor (FBR) at White Sands, the Army Pulsed Radiation Facility (Aberdeen, MD), and the Sandia Pulsed Reactor II used U--Mo alloys. All of these reactors utilized the \textgamma~phase of uranium, but because of the short irradiation time, the impacts of fuel burnup were minimal~\cite{horak1973operating}. The Dounreay Fast Reactor used a number of metal-fuel-based designs, which inclueds U--9.1Mo (9.1 wt\% Mo) and U--7Mo clad in niobium. The highly alloyed fuel cracked more, even though the 9.1-wt\%-Mo fuel swelled slightly less than 7 wt\%-Mo alloy~\cite{cottrell1964development}. In U.S. the Enrico Fermi Fast Breeder Reactor (EFFBR) was the first commercial fast reactor that used U--10Mo fuel. The primary concern was to ensure that the fuel would remain in the \textgamma-phase during the operating condition. A series of experiments have been performed to map the fission rate and temperature dependence of the \textgamma-phase's stability~\cite{no20031374}. Two types of U--Mo alloy fuel have been designed and tested. One is monolithic fuel, in which a thin layer of U--Mo foil is bonded to aluminum cladding. The other is a dispersion fuel in which of U--Mo fuel particles are dispersed in an Al matrix.

%\begin{figure}
%\centering
%\includegraphics[scale=1.0]{dispersion_monolithic.jpg}
%\caption[Schematic of monolithic and dispersion fuel]{Schematic diagram for monolithic and dispersion fuel from Jeong \etal~\cite{jeong2015mechanical}}
%\end{figure}
 

For five decades, dispersion fuels have powered many test and research reactors worldwide. The manufacturing process and operating conditions are well established for these types of fuels. The high-burnup testing of the dispersion fuel showed a pattern of \mbox{breakaway} swelling~\footnotemark\@ behavior at intermediate burnup. The post-irradiation examination of the U--Mo dispersion fuel revealed that this phenomenon is due to fission gas released from the interaction layer. Reaction between the U--Mo and aluminum occurs during irradiation and forms a ternary [(U--Mo)Al$_x$] phase which releases the fission gas at the boundary between the interaction phase and the aluminum matrix~\cite{leenaers2004post,jue2014microstructural,van2008transmission, olander2009growth}. These gas bubbles have a tendency to aggregate into the gas pockets, which weakens the fuel meat by exerting internal pressure. The result is mechanical failure and increase in fuel volume. To eliminate the fuel--matrix interaction, a `monolithic' U--Mo fuel was suggested. In monolithic fuels, a zirconium foil is used as a diffusion barrier between the fuel and the cladding (aluminum) to prevent diffusion of molybdenum into the cladding~\cite{jue2014microstructural}.

\footnotetext{The \textit{breakaway} is defined as "the limiting exposure, beyond which there will be a marked increase in the rate of swelling as a function of burnup~\cite{osti_10163384}"}

This dissertation investigates how fission gas (xenon and krypton) impacts the transport properties of U--Mo fuel. In Chapter~2 we discuss the reduction of thermal conductivity due to the presence of fission gas. In Chapter~3 we introduce a new pseudopotential for metallic uranium to study the properties using density functional theory (DFT). A basic background of DFT is also discussed in this chapter. In Chapter~4 we look into the diffusion of xenon in \textgamma-uranium and U--Mo alloy. In Chapter~5 we discuss the conclusions and future work.


%In the current work we have investigated how fission gas (xenon and krypton) impacts the U--Mo fuel. In the second chapter we have studied the reduction of thermal conductivity due to the presence of xenon gas. We have implemented finite element method to study the different microstructural configuration of fission gas in U--Mo fuel. In the third chapter we have introduced a new \textit{pseudopotential} for metallic uranium to study the properties using first-principles (Density Functional Theory (DFT)) method. In the fourth chapter we will discuss about the atomistic diffusion mechanism of xenon in U--Mo fuel, which includes the study of xenon gas in the U--Mo fuel using DFT methods. 
















\bibliographystyle{unsrt}
\bibliography{abbreviated,comp}
