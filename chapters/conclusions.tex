\chapter{Conclusions and Future Work}
We have studied the impact of fission gas on U--Mo alloys. Xenon and krypton gas bubbles inside the fuel have a significant impact on the thermal conductivities of the fuel. In comparison to pure xenon bubbles in U-10Mo, a xenon--krypton mixture makes very little differences in terms of thermal conductivity. This reaffirms that xenon contributes the most to the thermal conductivity. Both intra- and inter-granular xenon bubbles impact the heat transfer. The arrangements of the bubbles also influence the heat transfer. 


Since xenon forms a gas-bubble superlattice in U--Mo alloys, we can estimate the reduction of the thermal conductivity within reasonable precision. Grain boundaries also play an important role in the collection of fission gas. The growth of grain boundaries in U-10Mo should be studied. The current model can be adjusted to accommodate the expansion of grain boundaries. It is important to mention that, all of our studies here used a two-dimensional (2D) model. Mathematical models of thermal conductivity in 2D underestimate the three dimensional thermal conductivities. However, because of the complexities of studying the grain boundaries in three dimensions, one can get the general idea using a simpler 2D model as was done here. The growth of grain boundaries and recrystallization may change the grain structure.


Solid fission products play important roles in metallic fuel. The influence of intermetallic compounds has already been observed in the case of dispersion U--Mo fuel. We have not included solid fission products in our thermal conductivity model for U--Mo fuel, but this could be a possible improvement. They can influence both the swelling and transport processes inside the fuel. In monolithic fuel, there are two layers of cladding around the fuel meat. One is aluminum and other one is zirconium. These two layers will have significant impact on the heat transfer and should be studied extensively. A proper thermal conductivity model with contact pressure analysis would also improve the one discussed in Ch.~3.

We also introduced a new pseudopotential for density functional theory studies of uranium. The electronic properties of the three phases of metallic uranium were calculated and compared with previous results. The new pseudopotential shows that \textalpha-uranium has the lowest energy in the ground state. The elastic moduli for \textalpha-\@ and \textgamma-uranium were also calculated and compared. We used this model to study supercells of \textgamma-uranium and uranium--molybdenum alloys. The vacancy formation energy agrees well with both experiment and previous theoretical studies. 

Xenon forms inside the fuel matrix as a fission product. Because of its size, it uses vacancies to diffuse inside the matrix. We calculated the xenon migration energy from the lattice site to the nearest vacancy in both \textgamma-uranium and uranium--molybdenum alloys. In pure \textgamma-uranium, the migration energy is about 0.161 eV, which is lower than the molybdenum migration energy of 2.067 eV. This higher migration energy of molybdenum indicates that in \textgamma-uranium, molybdenum atoms move very slowly. Xenon moves significantly faster. In the presence of a vacancy, xenon tends to find its minimum-energy position in between the lattice site and the vacancy. 

Uranium--molybdenum alloys are disordered (\ie, non-stoichiometric). This makes \textit{ab initio} study of uranium--molybdenum alloys challenging. To study xenon diffusion inside uranium--molybdenum fuel, we systematically included molybdenum in nearest-neighbor locations of a bcc lattice. We did not include the effects of molybdenum in the second-nearest-neighbor shell, which could be a potential future study to see how the second shell impacts the migration energy of xenon. According to our calculations, the migration energy of xenon increases when a single molybdenum atom is present in a first-nearest-neighbor location. As the number of molybdenum atoms increases, the value of migration energy may also increase, depending on the direction of the migration and the presence of molybdenum at sites near the pathway. The overall trend is an increase of the migration energy with molybdenum content. To estimate the diffusion coefficient of xenon in uranium--molybdenum, a kinetic Monte Carlo simulation can be used that incorporates these different migration pathways.
 

Finally, we discussed the defect properties of lithium and its interactions with helium. Lithium, a potential plasma-facing material, is the first metal in the periodic table. The vacancy formation and migration energies are calculated and compared with previous theoretical work. The formation energies of helium in interstitial positions are calculated. The lowest-energy position is the tetrahedral position in bcc lithium. Our calculation predicts that the formation energy of helium in a substitutional position is higher than that of tetrahedral and octahedral interstitial locations.

The binding energies of helium to vacancies are also calculated. The binding energies of helium in tetrahedral locations and nearby vacancies are higher than the binding energy of helium to a lattice (substitutional) site. Multiple helium atoms bound to a vacancy have higher and higher binding energies as the number of helium increases. Because of the size of helium relative to lithium and the larger lattice parameter of lithium, helium can migrate from one interstitial site to another without a relatively high barrier. The lowest migration energy is 0.003 eV for the migration from a tetrahedral site to another tetrahedral site. Future work should study helium mobility through complex grain boundaries and surfaces in lithium. Molecular dynamics methods can be used to study grain boundaries. A robust pair potential for lithium--helium system is required, and needs to be developed. The DFT results in the current work can be used to fit a pair potential.




