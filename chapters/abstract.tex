\begin{abstract}
Inert gases cause some unique challenges in nuclear fuel development. In this dissertation, we study the impact of inert gases on the transport behavior in next-generation fission and fusion reactors.
Uranium--molybdenum alloys are potential nuclear fuels that can replace highly-enriched uranium fuel in research and test reactors around the world. Developing a uranium fuel with low enrichment is an important step towards preventing nuclear proliferation and promoting the peaceful use of nuclear energy\@. We look into how fission gas (\ie, krypton and xenon) impacts the transport properties of U--Mo alloys. We start with an analysis of the impact of fission gas on the overall thermal conductivity of \mbox{U--Mo}. We find that the presence of fission gas inside the fuel significantly reduces the overall thermal conductivity. We then discuss the electronic structure of uranium using density functional theory (DFT) and introduce a new pseudopotential for uranium. This pseudopotential is then used to examine xenon migration in U--Mo alloys. Our results show that the presence of molybdenum in nearest-neighbor lattice positions increases the migration energy of xenon relative to pure \textgamma-uranium, meaning molybdenum impedes fission gas transport. Finally, we study the interaction of helium with lithium, which is a potential plasma-facing material for fusion reactors, using DFT\@. We find that helium behaves very differently in lithium than it does in other bcc materials. Our studies show that helium has very low migration energies inside lithium, indicating high mobility.
\end{abstract}
