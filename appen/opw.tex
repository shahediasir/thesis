\chapter{Orthogonal Plane Wave}\label{appen_opw}

For now, we will try to approximate the time-independent \schrod equation so that we can achieve self-consistency. Let $V(r)$ be the potential seen by each electron. Then each energy eigenfunction will satisfy the following equation:
\begin{equation}
\label{eigeqn}
\hat{\mathcal{H}}\psi_i = \left[\hat{T}+\hat{V}(r)\right]\psi_i = E_i\psi_i 
\end{equation}
Here, $\hat{T}$ is the kinetic energy operator ($-\hbar^2\nabla^2/2m$), and $E_i$ is the energy of the $i$-th state. The next step is to distinguish between the core and the valence state. The index $\alpha$ will be used for the core electrons and $v$ will be used for the valence band. The core states are the same as in the isolated ion, but their energies are different (\ie, $E_\alpha$, is different):
\begin{equation}
\label{eq_eigalpha}
\left[\hat{T}+\hat{V}(r)\right]\psi_\alpha = E_\alpha \psi_\alpha
\end{equation}
Here, the subscript $\alpha$ not only denotes the position of the ion but also the energy and angular momentum quantum numbers of the state in the equation. This is a well-defined Sturm--Liouville eigenvalue problem, apart from not knowing $V(r)$. We can approach the problem in several ways. The most common is to represent $\psi_{\alpha}$ by a basis set. Plane-wave basis sets are often used in band structure calculations. If we expand the \schrod Eqn.~\eqref{eq_eigalpha} with a complete set of states,  then we may obtain a linear solution of simultaneous equations in the expansion coefficients. The problem of  solving differential equations then becomes a matrix diagonalization problem. 

The choice of plane waves has its pros and cons. One of the difficulties of using plane waves is that, it typically requires a large number of plane waves to give a reasonable description of the wavefunction. Thus the solution becomes very difficult. In addition the core electrons have higher kinetic energy near the nucleus (see Fig.~\ref{fig_hydrogen}), which makes it even harder for plane waves to approximate those electrons. Herring~\cite{herring1940new} suggested that, rather than expanding the valence electron wave function in plane waves, a more rapidly convergent procedure is to expand in \textit{orthogonalized} plane waves (OPWs). Hopefully, the expansions in terms of OPWs, would require fewer terms and therefore yield a faster calculation. The OPWs with wave number $\mathbf{k}$ can be defined as follows:
\begin{equation}
\label{eq_opw}
OPW_{\vb{k}} = e^{i\vb{k\cdot r}} - \sum_{\alpha} \psi_{\alpha}(\vb{r}) \psi^{\ast}_{\alpha} e^{i{\vb{k\cdot r}}} d \vb{r}
\end{equation}

An example would be sodium, which has a ground state configuration of $1s^2 2s^2 2p^6 3s^1$. The core orbitals would be $1s^2 2s^2 2p^6$. For so-called simple metals (\ie, Na, Mg, and Al), the electron wavefunction is rapidly convergent in the OPW basis. Before going any further, let's check whether this OPW (\ref{eq_opw}) is orthogonal to the core states. Let's assume an arbitrary core wavefunction $\psi_{\beta}$ and check the orthogonality with \ref{eq_opw}:
\begin{equation}
\int\psi^{\ast}_{\beta}(\vb{r})OPW_{\vb{k}}d\vb{r} = \int \psi^{\ast}_{\beta} (\vb{r}) e^{i\vb{k\cdot r}} d\vb{r} - \sum_{\alpha} \delta_{\alpha\beta} \int \psi^{\ast}_{\alpha} (\vb{r}) e^{i\vb{k\cdot r}} d\vb{r} = 0 
\end{equation}
It is convenient to normalize the plane waves in the unit cell volume of the metal $\Omega$, and we will use Dirac's bra-ket notation for the wave functions from now on. The plane wave becomes
\begin{equation}
	\ket{\vb{k}} = \Omega^{-1/2}e^{i\vb{k\cdot r}}
\end{equation}
For core electron wave functions
\begin{equation}
	\ket{\alpha} \equiv \psi_{\alpha} (\vb{r}) 
\end{equation}
where $\bra{\alpha} = \ket{\alpha}^{\ast}$, and
\begin{equation}
	\braket{\alpha}{\vb{k}} = \Omega^{-1/2} \int \psi^{\ast}_{\alpha} (\vb{r}) e^{i\vb{k\cdot r}} d\vb{r}
\end{equation}
Thus the OPW equations becomes
\begin{equation}
	OPW_{\vb{k}} = \ket{\vb{k}} - \sum_{\alpha} \ket{\alpha}\braket{\alpha}{\vb{k}}
\end{equation}
The projection operator $P$ can be defined as follows:
\begin{equation}
\label{eq_proj}
P = \sum_{\alpha} \dyad{\alpha}{\alpha}.
\end{equation}
The $P$ operator projects any function onto the core states. In terms of the $P$ operator the OPW can take following form:
\begin{equation}
	OPW_{\vb{k}} = (1-P)\ket{\vb{k}}.
\end{equation}
Now, we can expand the valence-band (valence electron) state as a linear combination of  OPWs:
\begin{equation}
	\psi_{k} = \sum_q a_q (\vb{k})(1-P)\ket{\vb{k}+\vb{q}}
\end{equation}
Before going a bit further, lets see how kinetic energy is represented by plane waves.
\begin{equation}
\label{eq_kin}
	\bra{q'}-\frac{\hbar^2}{2m}\nabla^2\ket{q} = \frac{1}{2}\frac{\hbar^2}{2m} \abs{\vb{q}}^2 \delta_{qq'}
\end{equation}
In the above approximation of kinetic energy we assumed the normalizing factor is one.
Now we are going to expand \ref{eq_eigalpha} with OPWs, and the \schrod equation becomes
\begin{equation}
\hat{\mathcal{H}}\psi_k = \sum_q a_q (\vb{k})\hat{\mathcal{H}}(1-P)\ket{\vb{k}+\vb{q}}=E_k\sum_q a_q (\vb{k}) (1-P) \ket{\vb{k}+\vb{q}}
\end{equation}
where $\hat{\mathcal{H}}$ consists both the kinetic energy and the potential energy. Multiplying on the left by $\bra{\vb{k}+\vb{q}'}$, and using Equation~\eqref{eq_kin} we obtain
\begin{multline}
\label{eq_opwd}
a_{q'} (\vb{k}) \frac{\hbar^2}{2m}\abs{\vb{k}+\vb{q}'}^2 + \sum_q a_q (\vb{k})[\bra{\vb{k}+\vb{q}'}V\ket{\vb{k}+\vb{q}} - \sum_{\alpha} E_{\alpha} \braket{\vb{k}+\vb{q}'}{\alpha}\braket{\alpha}{\vb{k}+\vb{q}}]
	= \\
 [a_{q'} (\vb{k}) - \sum_q a_q (\vb{k}) \bra{\vb{k}+\vb{q}'}P\ket{\vb{k}+\vb{q}}]E_{\vb{k}}  
\end{multline}
The above equation can be solved by diagonalizing some of matrix element. If we can evaluate some of the various matrix elements (integrals), we obtain a set of linear algebraic equations.


\bibliographystyle{apsrev4-2}
\bibliography{abbreviated,final}
%\bibliography{abbreviated,other} 

