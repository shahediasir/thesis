\chapter{Atomic Units}\label{appen_atomicunit}
There are special units of measurement which is convenient for atomic physics and DFT calculations. They are named after the physicist Douglas Hartree~\cite{hartree}. In this system of units, the numerical values of the four fundamental physical constants are assumed to be unity. They are as follows:
\begin{itemize}
\item Reduced Planck constant: $\hbar = 1$, the atomic unit of angular momentum.
\item Elementary charge: $e = 1$, the atomic unit of charge.
\item Bohr radius: $a_0 = 1$, atomic unit of length.
\item Electronic mass: $m_e = 1$, the atomic unit of mass.
\end{itemize}
Now we will explore how atomic units simplify quantum mechanical equations for the hydrogen atom. The hydrogen atom consists of a heavy proton (charge $e$) with a much lighter electron (charge $-e$) that orbits around it. The potential energy (in SI units) is
\begin{equation}\label{eq_hyd_pot}
V(\vb{r}) = -\frac{Ze^2}{4\pi \epsilon_0 r}
\end{equation}
where $Z$ is the atomic number and $r$ is the position relative to the nuclear site. The radial Schr\"odinger equation including the centrifugal term is:
\begin{equation}\label{eq_hdr_atom}
	\left [ -\frac{\hbar^2}{2m_e}\nabla_r^2 + \frac{\hbar^2 \ell (\ell+1)}{2m_er^2} - \frac{Ze^2}{4\pi \epsilon_0 r} - E \right ]rR(r) = 0
\end{equation}
This equation can be simplified by introducing dimensionless quantities. Multiplying Eq.~\eqref{eq_hdr_atom} by $m_e/\hbar^2$ the equation becomes
\begin{align}
\begin{split}
\left [-\nabla^2_r + \frac{\ell(\ell+1)}{r^2} + \frac{2m_eZe^2}{4\pi\epsilon_0\hbar^2} \frac{1}{r} - \frac{2m_eE}{\hbar^2} \right ] rR(r) & = 0 \\
\left [ -\frac{1}{2} \nabla_r^2 + \frac{\ell(\ell+1)}{2r^2} + \frac{Z}{a_0r} - \frac{E}{E_0a_0^2}    \right ]rR(r) & = 0 \\
\left [-\frac{1}{2}\nabla_{\zeta}^2 + \frac{\ell(\ell+1)}{2\zeta^2} + \frac{Z}{\zeta} - \frac{E}{E_0}    \right ] \zeta R(\zeta a_0) & = 0
\end{split}
\end{align} 
The above equation has two units: one is a length scale, the Bohr radius $a_0$ and the other is an energy scale, the hartree $E_0$.

\section{Bohr Radius}
\begin{equation}
\label{eq_bohr}
a_0 = \frac{4\pi\epsilon_0 \hbar^2}{m_ee^2}
\end{equation}
The Bohr radius is the length unit of the hartree atomic unit system. It corresponds to the radius of a classical electron circling a proton at the ground state energy of the hydrogen atom. Using all the fundamental constants in Eq.~\eqref{eq_bohr} the standard atomic unit of length is
\begin{equation}
1\ \text{bohr} = \frac{4\pi\epsilon_0 \hbar^2}{m_ee^2} = \num{0.52917725e-10}\ \text{m} = 0.52917725 \ \si{\angstrom}
\end{equation}


\section{Hartree}
\begin{equation}
\label{eq_htr}
 E_0= \frac{\hbar^2}{m_ea_0^2} = \frac{m_ee^4}{(4\pi\epsilon_0\hbar)^2}
\end{equation}
The Hartree is the energy unit in the Hartree atomic unit system. One Hartree is twice the binding energy of an electron in the hydrogen atom. To satisfy $a_0 = 1$, the fundamental constants are defined to be unity ($\hbar = m_e = e = 4\pi\epsilon_0 = 1$), according to Eq.~\eqref{eq_bohr}. The unit of Hartree can also be calculated from the fundamental constants:
\begin{equation}
1\ \text{hartree} = \frac{\hbar^2}{m_ea_0^2} = \frac{m_ee^4}{(4\pi\epsilon_0\hbar)^2} = \num{4.3597482e-10}~\mathrm{J} = 27.211396~\mathrm{eV}
\end{equation}
In Hartree units, the Schr\"odinger equation of an electron in the Coulomb potential of the nucleus has the following form:
\begin{equation}
\left(-\frac{1}{2}\nabla^2 - \frac{Z}{r} - E \right) \ket{\psi} = 0
\end{equation}

\section{Rydberg Units}
The Rydberg is another useful atomic unit which is widely used in DFT and atomic physics. In Hartree atomic units, the assumption is $\hbar^2/2m_e = \frac{1}{2}$. In Rydberg atomic units the equivalent assumption is: 
\begin{equation*}
\hbar = 2m_e = e^2/2 = 1
\end{equation*}
Using the above relations, the length unit does not change:
\begin{equation}
a_0 = \frac{4\pi\epsilon_0 \hbar^2}{2m_ee^2/2} = \frac{4\pi\epsilon_0 \hbar^2}{m_ee^2} = 0.52917725~\si{\angstrom}
\end{equation}
The energy unit does change:
\begin{equation}
1\ \text{Ry} =  \frac{\hbar^2}{2m_ea_0^2} = \frac{1}{2} \frac{\hbar^2}{m_ea_0^2} = 13.605693\  \text{eV} = \frac{1}{2}~\mathrm{Ha}
\end{equation}








\bibliographystyle{apsrev4-2}
\bibliography{abbreviated,final}
%\bibliography{abbreviated,other}
