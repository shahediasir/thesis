\chapter{Projector Augmented Wave}\label{appen_paw}

The pseudopotential technique has proven to be accurate for a large variety of systems, but there is no strict guarantee that it will produce the same results as an all-electron calculation. The challenge with norm-conserving pseudopotentials is that they limit the \textit{softness}. Another way to say it is that, it requires a high cut-off energy. In the plane-wave basis set for the pseudo wavefunctions is defined by the shortest wave length $\lambda=2\pi/\abs{\vb{G}}$, where $\vb{G}$ is the wave vector. Projector augmented waves (PAW) introduces projectors and auxiliary localized functions to increase the softness of the pseudopotential, while at the same time keeping the full wavefunction. In this section, I will try to introduce PAW with basic formalism. The origin of the PAW method lies in a transformation that maps the true wavefunctions with their complete nodal structure onto auxiliary wavefunctions. The purpose of this transformation is to have smooth auxiliary wavefunctions that have a rapidly convergent plane-wave expansion. The PAW method was first proposed by Bl\"ochl in 1994~\cite{blochl1994projector}. The linear transformation is as follows:
\begin{equation}
\ket{\Psi_n} = \hat{\mathcal{T}}\ket{\tilde{\Psi}_n}
\end{equation}
where, $\ket{\Psi_n}$ is the true all-electron Kohn--Sham (KS) single-particle wavefunction, $\ket{\tilde{\Psi}_n}$ is an auxiliary smooth wavefunction, and $\hat{\mathcal{T}}$ is a linear transformation operator. Since the true wavefunctions are already smooth at a certain minimum distance from the core, $\tilde{\mathcal{T}}$ should only modify the wavefunction close to the nuclei. Thus the transformation operator becomes
\begin{equation}
\tilde{\mathcal{T}} = 1 + \sum_a \tilde{\mathcal{T}^a},
\end{equation}
where $a$ is an atom index and $\tilde{\mathcal{T}^a}$ has no effect outside a certain atom-specific augmentation region, $r_C^a$. Inside the augmentation spheres, the true wavefunction can be expanded in the partial waves $\phi_i^a$, for a corresponding auxiliary smooth partial wave can be defined as $\tilde{\phi}_i^a$, and they can be connected by the following relation:
\begin{equation}
\label{eq_aug_t}
\ket{\phi_i^a} = (1+\hat{\mathcal{T}}^a)\ket{\tilde{\phi}_i^a} \Rightarrow  \hat{\mathcal{T}}^a \ket{\phi_i^a} = \ket{\phi_i^a} - \ket{\tilde{\phi}_i^a}.
\end{equation} 
Here $a$ is an atom index and $i$ denotes partial waves. Outside the augmentation sphere, the partial wave and its smooth counterpart should be identical:
\begin{equation}
\phi_i^a(\mathbf{r}) = \tilde{\phi}_i^a(\mathbf{r})\quad \text{for} \quad r > r_c^a
\end{equation} 
Where $\phi_i^a(\mathbf{r})=\braket{\mathbf{r}}{\phi_i^a}$, and similar for $\tilde{\phi}_i^a$. If the smooth partial waves form a complete set inside the augmentation sphere, we can expand the smooth all-electron wavefunctions as 
\begin{equation}
\ket{\tilde{\Psi}_i^a} = \sum_i P_{ni}^a \ket{\tilde{\phi}_i^a} \quad \abs{\mathbf{r}-\mathbf{R}} < r_c^a
\end{equation}
where, $P_{ni}^a$ are expansion coefficients, that need to be determined. The index $a$ stands for atomic sites, $i$ to dintinguish different partials waves,  and $n$ is the principle to quantum numbers $(\ell,m)$. The transformation operator connects the smooth pseudo wavefunction to true wavefunction.
\begin{equation}
\ket{\Psi_n} = \hat{\mathcal{T}} \ket{\tilde{\Psi}_n} = \sum_i P_{ni}^a \ket{\psi_i^a} \quad \abs{\mathbf{r}-\mathbf{R}} < r_c^a
\end{equation} 
Something really interesting about the above equation is that the true wavefunction has the same expansion coefficient $(P_{ni}^a)$ as the pseudo-wavefucntion. The transformation operator $\hat{\mathcal{T}}$ is required to be linear, the coefficient must be linear functionals of $\ket{\tilde{\Psi}_n}$, \ie,
\begin{equation}
P_{ni}^a = \braket{\tilde{p}_i^a}{\tilde{\Psi}_n}
\end{equation}
where $\ket{\tilde{p}_i^a}$ are some fixed functions termed smooth projector functions. As there is no overlap between the augmentation spheres, we expect the smooth all-electron wavefunction, $\ket{\tilde{\Psi}_n^a} = \sum_i \ket{\tilde{\phi}_i^a} \braket{\tilde{p}_i^a}{\tilde{\Psi}_n}$. The projectors have to be localized within an augmentation region, so
\begin{equation}
\label{eq_paw_complete}
	\sum_i \dyad{\tilde{\phi}_i^a}{\tilde{p}_i^a} = 1.
\end{equation}
This also implied that
\begin{equation}
	\braket{\tilde{p}_{i_1}^a}{\tilde{\phi}_{i_2}^a} = \delta_{i_1,i_2}\quad \text{for} \quad r < r_c^a
\end{equation}
the projector functions should be orthonormal to the smooth partial waves inside the augmentation sphere. The choice of projectors and partial waves can be found more detailed form original work of Bl\"ochl~\cite{blochl1994projector}. Using the completeness relation from Eqn.~\eqref{eq_paw_complete}
\begin{equation}
\hat{\mathcal{T}^a} = \sum_i \hat{\mathcal{T}^a} \dyad{\tilde{\phi}_i^a}{\tilde{p}_i^a} = \sum_i \left( \ket{\phi_i^a} -\ket{\tilde{\phi}_i^a} \right) \bra{\tilde{p}_i^a}.
\end{equation}
Remember, the operator $\hat{\mathcal{T}}^a$ operates only inside the sphere; outside sphere, it behaves like $\ket{\phi_i^a} - \ket{\tilde{\phi}_i^a}$. Thus, the total transformation operator becomes
\begin{equation}
\hat{\mathcal{T}} = 1 + \sum_a\sum_i\left( \ket{\phi_i^a} -\ket{\tilde{\phi}_i^a} \right) \bra{\tilde{p}_i^a}.
\end{equation}
To summarize, we obtain the all-electron KS wavefunction $\ket{\Psi_n(\mathbf{r})} = \braket{\mathbf{r}}{\Psi_n}$ from the transformation
\begin{equation}
\label{eq_main_paw}
\Psi_n(\mathbf{r}) = \tilde{\Psi}_n(\mathbf{r}) + \sum_a\sum_i \left (\phi_i^a(\mathbf{r}) - \tilde{\phi}_i^a(\mathbf{r})   \right) \braket{\tilde{p}_i^a}{\tilde{\psi}_n}
\end{equation}
The equation above has three different components on the right. The first is the auxiliary wavefunction. The second term is the sum of partial waves, the last term is the sum of pseudo-partial waves that must be subtracted inside the augmentation region. To make it a simple KS wavefunction representation
\begin{equation}
\psi_n(\mathbf{r}) = \tilde{\psi}_n(\mathbf{r}) + \sum_a\left( \psi_n^a(\mathbf{r}-\mathbf{R}^a) - \tilde{\psi}_n^a (\mathbf{r} - \mathbf{R}^a) \right).
\end{equation}
The trouble with the original KS wavefunction was that they display oscillations near the nucleus and smooth behavior away from the nucleus. By decomposing the wavefunction in the manner of Eqn.~\eqref{eq_main_paw}, the achievement is that the original wavefunction is separated into auxiliary wavefunctions that are smooth everywhere.

\clearpage


\section{PAW Pseudopotential Generation}\label{appen_pseudo}
The PAW calculation method requires a set of basis functions (partial-waves) and projector functions as well as some additional atomic data in the so called \textit{PAW dataset}. The PAW dataset is generated using the following procedure:

\begin{itemize}
	\item Solve the all-electron atomic problem in the DFT formalism using an exchange--correlation functional (with the scalar-relativistic approximation). It is a spherical problem and usually solved with a logarithmic grid.
	\item Separation of core and valence electrons. The core density is then deduced from the core electron wavefunctions. For a given radius ($r_{\text{core}}$), the outer density is calculated so that the core density is identical to the outer.
	\item Choose the PAW basis (number of partial waves and projectors).
	\item Generation of the pseudo partial-waves.
	\item Test the PAW dataset on an electronic structure calculation. Repeat the procedure if necessary to match the calculated property's agreement with experiment or with all-electron calculations.
\end{itemize}


\bibliographystyle{apsrev4-2}
\bibliography{abbreviated,final}
\pagebreak
The atompaw input file used to generate the pseudopotential for uranium in Ch.~4 is below:
\lstset{style=atpw}
\begin{lstlisting}
U 92
GGA-PBE	scalarrelativistic loggridv4 700 
7 6 6 5 0 !6s2 6p6 5f3 6d1 7s2 (from Beelar (2012), ns_max np_max nd_max nf_max ng_max
6 2 1	!6d1, only empty or partially occupied shells are entered
5 3 3	!5f3
0 0 0
c	!1s2	1	
c	!2s2	2
c	!3s2	3
c	!4s2	4
c	!5s2	5
v	!6s2	6		1
v	!7s2	7		2
c	!2p6	8
c	!3p6	9
c	!4p6	10
c	!5p6	11		
v	!6p6	12		3
c	!3d10	13
c	!4d10	14
c	!5d10	15
v	!6d1	16		4
c	!4f14	17
v	!5f3	18		5
3
2.5 2.02 1.5 1.8	!rpaw, rshape, rvloc, rcore, in a.u , changesd the rshape for the first time below 2.0
n
y				!one additional p partial wave
0.00
n				!no additional p partial wave
y				!one additional d partial wave
0.2				!
n				!no additional d partial wave
y				!one additional f partial wave
3.0
n				!no additional f partial wave
custom  rrkj besselshape    !for PWscf, UPF format needs bessels        
4 0.0 !bessel     			!l quantum number, reference energy (Ry), bessel for simplicity
1.5		!1 s			! r_c matching radius for second s partial wave
1.5		!2 s			! rc for s
2.5		!3 p			! rc for p
2.5		!4 p			! rc for p
2.5		!5 d	        ! rc for d, 1.3 gives very good result
2.5		!6 d        ! rc for  d,	1.3 gives very good result, 4.0 energy
2.5		!7 f        ! rc for f
2.5		!8 f        ! rc for  f
PWSCFOUT
default
0

\end{lstlisting}


