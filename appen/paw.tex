\chapter{Projector Augmented Wave}\label{appen_paw}

Pseudopotential technique has been proven to be accurate for a large variety of systems, but there is no strict gurantee that it will produce the same results as an all-electron calculation. The challenge with norm-conserving pseudopotentials is that it limits the \textit{softness}. Another way to say it, it needs higher cut-off energy. In the plane-wave basis set for the pseudo wavefunctions is defined by the shortest wave lentgh $\lambda=2\pi/\abs{\vec{G}}$, where $\vec{G}$ is the wave vector. Projector augmented waves (PAW) introduces projectors and auxiliary localized functions to increase the softness of the pseudopotential, at the same time keeping full wave function. In this section I will try to introduce PAW with some basic formalism. The origin of the PAW method lies a transformation that maps the true wavefunctions with their complete nodal structure into auxiliary wavefunctions. The purpose of this transformation is to have a smooth auxiliary wavefunctions, which have a rapidly convergent plane-wave expansion. The PAW method was first proposed by Bl\"ochl in 1994~\cite{blochl1994projector}. The linear transformation is as follows.
\begin{equation}
\ket{\Psi_n} = \hat{\mathcal{T}}\ket{\tilde{\Psi}_n}
\end{equation}
Where, $\ket{\Psi_n}$ is the true all-electron KS single particle wave function, $\ket{\tilde{\Psi}_n}$ is auxiliary smooth wave function, and $\hat{\mathcal{T}}$ is a linear transformation operator. Since the true wave functions are already smooth at a certain minimum distance from the core, $\tilde{\mathcal{T}}$ should only modify the wave function close to the nuclei. Thus the transformation operator becomes
\begin{equation}
\tilde{\mathcal{T}} = 1 + \sum_a \tilde{\mathcal{T}^a}
\end{equation}
Where a is an atom index, $\tilde{\mathcal{T}^a}$, has no effect outside a certain atom-specific augmentation region, $r_C^a$. Inside the augmentation spheres, the true wave function can be expanded in the partial waves $\phi_i^a$, for a corresponding auxiliary smooth partial wave can be defined as $\tilde{\phi}_i^a$, and they connected by the following relation.
\begin{equation}
\label{eq_aug_t}
\ket{\phi_i^a} = (1+\hat{\mathcal{T}}^a)\ket{\tilde{\phi}_i^a} \Rightarrow  \hat{\mathcal{T}}^a \ket{\phi_i^a} = \ket{\phi_i^a} - \ket{\tilde{\phi}_i^a}
\end{equation} 
Here $a$ is an atom index, $i$ comes from partial waves. Outside the augmentation sphere, the partial wave and its smooth counterpart should be identical.
\begin{equation}
\phi_i^a(\mathbf{r}) = \tilde{\phi}_i^a(\mathbf{r}),\  for \quad r > r_c^a
\end{equation} 
Where $\phi_i^a(\mathbf{r})=\braket{\mathbf{r}}{\phi_i^a}$, and similar for $\tilde{\phi}_i^a$. If the smooth partial waves form a complete set inside the augmentation sphere, we can expand the smooth all electron wave functions as 
\begin{equation}
\ket{\tilde{\Psi}_i^a} = \sum_i P_{ni}^a \ket{\tilde{\phi}_i^a}, \quad \abs{\mathbf{r}-\mathbf{R}} < r_c^a
\end{equation}
Where, $P_{na}^a$ are expansion coefficients, and need to be determined. And also, the index a stands for atomic sites, i to dintinguish different partials waves and n connected to quantum numbers $(l,m)$. Since the transformation operator connects the smooth pseudo wave function to true wave function.
\begin{equation}
\ket{\Psi_n} = \hat{\mathcal{T}} \ket{\tilde{\Psi}_n} = \sum_i P_{ni}^a \ket{\psi_i^a}, \quad \abs{\mathbf{r}-\mathbf{R}} < r_c^a
\end{equation} 
Something really interesting about the above equation, the true wave function has the same expansion coefficient $(P_{ni}^a)$ as the pseudo wavefucntion. The transformation operator $\hat{\mathcal{T}}$ is required to be linear, the coefficient must be linear functionals of $\ket{\tilde{\Psi}_n}$, i.e.
\begin{equation}
P_{ni}^a = \braket{\tilde{p}_i^a}{\tilde{\Psi}_n}
\end{equation}
Where $\ket{\tilde{p}_i^a}$ are some fixed functions termed smooth projector functions. As there is no overlap between the augmentation spheres, we expect the smooth all electron wave function, $\ket{\tilde{\Psi}_n^a} = \sum_i \ket{\tilde{\phi}_i^a} \braket{\tilde{p}_i^a}{\tilde{\Psi}_n}$. The projectors have to be localized within an augmentation region
\begin{equation}
\label{eq_paw_complete}
	\sum_i \dyad{\tilde{\phi}_i^a}{\tilde{p}_i^a} = 1
\end{equation}
This also implied that
\begin{equation}
	\braket{\tilde{p}_{i_1}^a}{\tilde{\phi}_{i_2}^a} = \delta_{i_1,i_2},\quad for \quad < r_c^a
\end{equation}
the projector functions should be orthonormal to the smooth partial waves inside the augmentation sphere. The choice of projectors and partial waves can be found more detailed form original work of Bl\"ochl~\cite{blochl1994projector}. Using the completeness relation from Eqn~\ref{eq_paw_complete}
\begin{equation}
\hat{\mathcal{T}^a} = \sum_i \hat{\mathcal{T}^a} \dyad{\tilde{\phi}_i^a}{\tilde{p}_i^a} = \sum_i \left( \ket{\phi_i^a} -\ket{\tilde{\phi}_i^a} \right) \bra{\tilde{p}_i^a}
\end{equation}
Remember, the operator $\hat{\mathcal{T}}^a$ operates only inside the sphere, outside sphere it behave like this $\ket{\phi_i^a} - \ket{\tilde{\phi}_i^a}$. Thus the total transformation operator becomes
\begin{equation}
\hat{\mathcal{T}} = 1 + \sum_a\sum_i\left( \ket{\phi_i^a} -\ket{\tilde{\phi}_i^a} \right) \bra{\tilde{p}_i^a} 
\end{equation}
To summarize, we obtain the all electron KS wave function $\ket{\Psi_n(\mathbf{r})} = \braket{\mathbf{r}}{\Psi_n}$ from the transformation
\begin{equation}
\label{eq_main_paw}
\Psi_n(\mathbf{r}) = \tilde{\Psi}_n(\mathbf{r}) + \sum_a\sum_i \left (\phi_i^a(\mathbf{r}) - \tilde{\phi}_i^a(\mathbf{r})   \right) \braket{\tilde{p}_i^a}{\tilde{\psi}_n}
\end{equation}
The equation above has three different components in the right. The first is the auxiliary wave function. The second term is the sum of partial waves, the last term is the sum of pseudo partial waves that must be subtracted inside the augmentation region. To make it a simple KS wave function representation
\begin{equation}
\psi_n(\mathbf{r}) = \tilde{\psi}_n(\mathbf{r}) + \sum_a\left( \psi_n^a(\mathbf{r}-\mathbf{R}^a) - \tilde{\psi}_n^a (\mathbf{r} - \mathbf{R}^a) \right)
\end{equation}

The trouble of the original KS wave functions, was that they display rapid oscillations in some of the space and smooth behavior in other parts of space. By decomposing the wave function, the achievement is that the original wave function is separated into auxiliary wave functions which are smooth everywhere.


\section{PAW Pseudopotential Generation}\label{appen_pseudo}
PAW calculation method requires a set of basis (partial-waves) and projectors functions and some additional atomic data in the so called \textit{PAW dataset}. The PAW dataset generation is done using the following procedure.

\begin{itemize}
	\item Solve the all-electron atomic problem in the DFT formalism using an exchange-corrleation functional (with scalar-rlativistic approximation). It is a spherical problem and usually solved in logarithmic grid.
	\item Separation of core and valance electron. The core density is then deduced from the core electron wave functions. For a given radius ($r_{core}$), the outer density is calculated so that the core density is identical to the outer.
	\item Choose the PAW basis (number of partial-waves and projectors).
	\item Generation of the pseudo partial-waves.
	\item Test the PAW dataset on electronic structure calculation. Repeat the procedure if necessary to satisfy the calculated property.
\end{itemize}
\pagebreak
The atompaw input file used to generate the pseudopotential for uranium is below:
\lstset{style=atpw}
\begin{lstlisting}
U 92
GGA-PBE	scalarrelativistic loggridv4 700 
7 6 6 5 0 !6s2 6p6 5f3 6d1 7s2 (from Beelar (2012), ns_max np_max nd_max nf_max ng_max
6 2 1	!6d1, only empty or partially occupied shells are entered
5 3 3	!5f3
0 0 0
c	!1s2	1	
c	!2s2	2
c	!3s2	3
c	!4s2	4
c	!5s2	5
v	!6s2	6		1
v	!7s2	7		2
c	!2p6	8
c	!3p6	9
c	!4p6	10
c	!5p6	11		
v	!6p6	12		3
c	!3d10	13
c	!4d10	14
c	!5d10	15
v	!6d1	16		4
c	!4f14	17
v	!5f3	18		5
3
2.5 2.02 1.5 1.8	!rpaw, rshape, rvloc, rcore, in a.u , changesd the rshape for the first time below 2.0
n
y				!one additional p partial wave
0.00
n				!no additional p partial wave
y				!one additional d partial wave
0.2				!
n				!no additional d partial wave
y				!one additional f partial wave
3.0
n				!no additional f partial wave
custom  rrkj besselshape    !for PWscf, UPF format needs bessels        
4 0.0 !bessel     			!l quantum number, reference energy (Ry), bessel for simplicity
1.5		!1 s			! r_c matching radius for second s partial wave
1.5		!2 s			! rc for s
2.5		!3 p			! rc for p
2.5		!4 p			! rc for p
2.5		!5 d	        ! rc for d, 1.3 gives very good result
2.5		!6 d        ! rc for  d,	1.3 gives very good result, 4.0 energy
2.5		!7 f        ! rc for f
2.5		!8 f        ! rc for  f
PWSCFOUT
default
0

\end{lstlisting}


\bibliography{abbreviated,comp}
\bibliographystyle{iopart-num}
