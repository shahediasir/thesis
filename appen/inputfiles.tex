\chapter{Input Files}\label{appen_inputfiles}
Input files for different software are included in this chapter. Because of the complexity and number, I have included samples of them.

\section{MOOSE Framework}
This is the \texttt{main.C} file for moose simulation. Depends on the problem, moose requires a number of header files and source file.  

\lstset{style=cpp}
\begin{lstlisting}
  /* this is a main app file created for the purpose of simulating heat conduction
   *      This is file is created by Rafi, Sepetember 27, 2016
   *      *****************************************************   */
  
  #include "HeatCondRafiApp.h"
  #include "MooseInit.h"
  #include "Moose.h"
  #include "MooseApp.h"
  #include "AppFactory.h"
  
  // Creat a performance log
  PerfLog Moose::perf_log("HeatCondRafi");
  
  //Begin the main problem.
  int main(int argc, char *argv[])
  {
      // Initialize MPI, solvers and MOOSE
      MooseInit init(argc, argv);
  
      // Register this application's MooseApp and any it depends on
      HeatCondRafiApp::registerApps();
  
      // This creates dynamic memory that we're responsible for deleting
      //MooseApp * app =  AppFactory::createAppShared("HeatCondRafiApp", argc, argv);
    std::shared_ptr<MooseApp> app = AppFactory::createAppShared("HeatCondRafiApp", argc, argv);
  
      //Execute the application
      app->run();
  
      // Free up the memory we created earlier
      //delete app;
  
      return 0;
  }
\end{lstlisting}
\pagebreak
This is a sample of source file (kernel in MOOSE) of heat transfer in U--10Mo 
\lstset{style=cpp}
\begin{lstlisting}
// this file is edited by rafi fro temperature dependent thermal conductivity
#include "Tempdepk.h"

template<>
InputParameters validParams<Tempdepk>()
{
  InputParameters params = validParams<Diffusion>();
  params.addClassDescription("This will solve heat diffusion with temperature dependent thermal conductivity");
  return params;
}

Tempdepk::Tempdepk(const InputParameters & parameters) :
    Diffusion(parameters),
	_thermal_conductivity(getMaterialProperty<Real>("thermal_conductivity"))  /*be sure u use the "thermal_conductivity, otherwise it will
not read the thermal_conductivity from the input*/
{
}

Real
Tempdepk::computeQpResidual()
{
  //return _grad_u[_qp] * _grad_test[_i][_qp];
  //Real _k=0.606+0.0351*_u[_qp]; // from D.E. Burkes et al./ Journal of Nuc Material 2010

  return _thermal_conductivity[_qp]*Diffusion::computeQpResidual();
}

Real
Tempdepk::computeQpJacobian()
{
  //Real _k=0.606+0.0351*_u[_qp];
  //return _grad_phi[_j][_qp] * _grad_test[_i][_qp];

  return _thermal_conductivity[_qp]*Diffusion::computeQpJacobian();
}

\end{lstlisting}
\pagebreak

\lstset{style=cpp}
\begin{lstlisting}
// this file is edited by rafi fro temperature dependent thermal conductivity
#include "TXenon.h"


template<>
InputParameters validParams<TXenon>()
{
  InputParameters params = validParams<Diffusion>();
  params.addClassDescription("This will solve heat diffusion with temperature dependent thermal conductivity");
  return params;
}

TXenon::TXenon(const InputParameters & parameters) :
    Diffusion(parameters),
	_thermal_conductivity(getMaterialProperty<Real>("thermal_conductivity"))  /*be sure u use the "thermal_conductivity, otherwise it will
not read the thermal_conductivity from the input*/
{
}

Real
TXenon::computeQpResidual()
{
  //return _grad_u[_qp] * _grad_test[_i][_qp];
  //Real _k=0.606+0.0351*_u[_qp]; // from D.E. Burkes et al./ Journal of Nuc Material 2010

/* return _thermal_conductivity[_qp]*Diffusion::computeQpResidual() + 380E3*_test[_i][_qp]; */

// the above one for a constant forcing function

  return _thermal_conductivity[_qp]*Diffusion::computeQpResidual();
}

Real
TXenon::computeQpJacobian()
{
  //Real _k=0.606+0.0351*_u[_qp];
  //return _grad_phi[_j][_qp] * _grad_test[_i][_qp];

  return _thermal_conductivity[_qp]*Diffusion::computeQpJacobian() + 
	(-2*4.72984E-09*_u[_qp]*2.+2.0891E-5)*_grad_u[_qp]*_grad_test[_i][_qp];
}

\end{lstlisting}
\pagebreak
Material source code for U--10Mo property.
\lstset{style=cpp}
\begin{lstlisting}
// this file is edited by Rafi January 24, 2017
// 
#include "Tconductivity.h"

template<>
InputParameters validParams<Tconductivity>()
{
  InputParameters params = validParams<Material>();
// we do not need to add any data from the input file
  params.addClassDescription("This is to calculate the temperature dependent thermal conductivity of U-10Mo");
// the "temperature" variable has to be defined in the input files
  params.addCoupledVar("dep_variable", "The thermal conductivity is calculated from temperature");
  return params;
}

Tconductivity::Tconductivity(const InputParameters & parameters) :
    Material(parameters),

    // Declare  material properties.  This returns references that we
    // hold onto as member variables
    //_permeability(declareProperty<Real>("permeability")),
	_thermal_conductivity(declareProperty<Real>("thermal_conductivity")),
	//_dep_variable(isCoupled("dep_varible"))
	_dep_variable(coupledValue("dep_variable"))
{
}

void
Tconductivity::computeQpProperties()
{

  // Sample the LinearInterpolation object to get the permeability for the ball size
  //_permeability[_qp] = _permeability_interpolation.sample(_ball_radius);
  _thermal_conductivity[_qp]=0.606+0.0351*_dep_variable[_qp];
}
\end{lstlisting}
\pagebreak
\lstset{style=cpp}
Source file for xenon material property.
\begin{lstlisting}
// this file is edited by Rafi February 13, 2017
// 
#include "XenonT.h"

template<>
InputParameters validParams<XenonT>()
{
  InputParameters params = validParams<Material>();
// we do not need to add any data from the input file
  params.addClassDescription("This is to calculate the temperature dependent  thermal conductivity of Xenon");
// the "temperature" variable has to be defined in the input files
  params.addCoupledVar("dep_variable", "The thermal conductivity is calculated from temperature");
  return params;
}


XenonT::XenonT(const InputParameters & parameters) :
    Material(parameters),

    // Declare  material properties.  This returns references that we
    // hold onto as member variables
    //_permeability(declareProperty<Real>("permeability")),
	_thermal_conductivity(declareProperty<Real>("thermal_conductivity")),
	//_dep_variable(isCoupled("dep_varible"))
	_dep_variable(coupledValue("dep_variable"))
{
}

void
XenonT::computeQpProperties()
{

  // Sample the LinearInterpolation object to get the permeability for the ball size
  //_permeability[_qp] = _permeability_interpolation.sample(_ball_radius);
  //_thermal_conductivity[_qp]=0.000000001+0.00000000001*_dep_variable[_qp];
  /*abobe correlation is wrong, is calculated wrong, the next equation is calcuated from Robinovich et al. Thermophy
	sical property of Neon, Argon, Krypton, and Xenon */

	_thermal_conductivity[_qp]=-4.72984E-09*_dep_variable[_qp]*_dep_variable[_qp] + 2.0891E-5*_dep_variable[_qp]+ 2.83137E-05 ;
	 
	
}
\end{lstlisting}
\pagebreak

Source file for heat flux calculation.
\lstset{style=cpp}
\begin{lstlisting}

#include "HeatFlux.h"

template<>
InputParameters validParams<HeatFlux>()
{
  InputParameters params = validParams<AuxKernel>();

  MooseEnum component("x y z");
  // Declare the options for a MooseEnum.
  // These options will be presented to the user in Peacock
  // and if something other than these options is in the input file
  // an error will be printed
  params.addClassDescription("This is to calculate the heat flux for over all material");
  // Add a "coupling paramater" to get a variable from the input file.
  params.addRequiredCoupledVar("field_temperature", "The temperature field.");
  params.addRequiredParam<MooseEnum>("component", component, "The desired component of temperature");

  return params;
}

HeatFlux::HeatFlux(const InputParameters & parameters) :
    AuxKernel(parameters),

    // Get the gradient of the variable
    _temperature_gradient(coupledGradient("field_temperature")),

    // Snag thermal_coductivity from the Material system.
    // Only AuxKernels operating on Elemental Auxiliary Variables can do this
    _t_thermal_conductivity(getMaterialProperty<Real>("thermal_conductivity")),
	_component(getParam<MooseEnum>("component"))

{
}

Real
HeatFlux::computeValue()
{
  // Access the gradient of the pressure at this quadrature point
  // Then pull out the "component" of it we are looking for (x, y or z)
  // Note that getting a particular component of a gradient is done using the
  // parenthesis operator
  return _t_thermal_conductivity[_qp]*_temperature_gradient[_qp](_component);
}
\end{lstlisting}
This source file is to calculate the gradient of the temperature.
\lstset{style=cpp}
\begin{lstlisting}

#include "TGradient.h"

template<>
InputParameters validParams<TGradient>()
{
  InputParameters params = validParams<AuxKernel>();

  // Declare the options for a MooseEnum.
  // These options will be presented to the user in Peacock
  // and if something other than these options is in the input file
  // an error will be printed
  params.addClassDescription("This is to calculate the temperature gradient for over all material");
  // Add a "coupling paramater" to get a variable from the input file.
  //params.addRequiredCoupledVar("field_temperature", "The temperature field.");
  MooseEnum component("x y z");

  // Use the MooseEnum to add a parameter called "component"
  params.addRequiredCoupledVar("field_temperature", "The Temperature field");
  params.addRequiredParam<MooseEnum>("component", component, "The desired component of velocity.");
  return params;
}

TGradient::TGradient(const InputParameters & parameters) :
    AuxKernel(parameters),

    // Get the gradient of the variable
    _temperature_gradient(coupledGradient("field_temperature")),

	_component(getParam<MooseEnum>("component"))
    // Snag thermal_coductivity from the Material system.
    // Only AuxKernels operating on Elemental Auxiliary Variables can do this
    //_t_thermal_conductivity(getMaterialProperty<Real>("thermal_conductivity"))

{
}

Real 
TGradient::computeValue()
{
  // Access the gradient of the pressure at this quadrature point
  // Then pull out the "component" of it we are looking for (x, y or z)
  // Note that getting a particular component of a gradient is done using the
  // parenthesis operator
  return _temperature_gradient[_qp](_component);
}
\end{lstlisting}
\pagebreak
For every source file, we need header files. The required header files are below.
\texttt{Tempdepk.h}
\lstset{style=cpp}
\begin{lstlisting}
// this file created by Rafi, January 24, 2017
#ifndef TEMPDEPK_H
#define TEMPDEPK_H
#include "Diffusion.h"
class Tempdepk;
template<>
InputParameters validParams<Tempdepk>();

/**
 * This kernel will solve a temperature dependent thermal conductivity with diffusion of temperature
 */
class Tempdepk : public Diffusion
{
public:
  Tempdepk(const InputParameters & parameters);
protected:
  virtual Real computeQpResidual() override;
  virtual Real computeQpJacobian() override;
// creating a moose array for material property
 const MaterialProperty<Real> & _thermal_conductivity;
};
#endif 

\end{lstlisting}
\pagebreak
\texttt{TXenon.h}
\lstset{style=cpp}
\begin{lstlisting}
/ this file created by Rafi, January 24, 2017
#ifndef TXENON_H
#define TXENON_H

#include "Diffusion.h"

class TXenon;
template<>
InputParameters validParams<TXenon>();

/**
 * This kernel will solve a temperature dependent thermal conductivity with diffusion of temperature
 */
class TXenon : public Diffusion
{
public:
  TXenon(const InputParameters & parameters);

protected:
  virtual Real computeQpResidual() override;

  virtual Real computeQpJacobian() override;
// creating a moose array for material property

 const MaterialProperty<Real> & _thermal_conductivity;
};
#endif 

\end{lstlisting}
\pagebreak
\lstset{style=cpp}
\begin{lstlisting}
// this file is created by Rafi at February 13, 2017

#ifndef TCONDUCTIVITY_H
#define TCONDUCTIVITY_H

#include "Material.h"

class Tconductivity;

template<>
InputParameters validParams<Tconductivity>();

/**
 * Material objects inherit from Material and override computeQpProperties.
 *
 * Their job is to declare properties for use by other objects in the
 * calculation such as Kernels and BoundaryConditions.
 */
class Tconductivity : public Material
{
public:
  Tconductivity(const InputParameters & parameters);

protected:
  /**
   * Necessary override.  This is where the values of the properties
   * are computed.
   */
  virtual void computeQpProperties() override;

private:
    // The thermal conductivity 
  MaterialProperty<Real> & _thermal_conductivity; // for u-10 Mo

  // const VariableGradient & _dep_variable; // this will get the temperature 
  //std::vector<const VariableValue> & _dep_variable;
  //VariableValue<Real> & _dep_variable;
   const VariableValue & _dep_variable;
};
#endif //Tconductivity
\end{lstlisting}
\pagebreak
\texttt{XenonT.h}
\lstset{style=cpp}
\begin{lstlisting}
/ this file is created by Rafi at February 13, 2017
// The reference is Rabinovich et al.
// Thermophysical Propeties of Neon, Argon, Krypton and Xenon// for the purpose of temperature dependent thermal condcutivity for U-10Mo
#ifndef XENONT_H
#define XENONT_H

#include "Material.h"
class XenonT;

template<>
InputParameters validParams<XenonT>();

/**
 * Material objects inherit from Material and override computeQpProperties.
 *
 * Their job is to declare properties for use by other objects in the
 * calculation such as Kernels and BoundaryConditions.
 */
class XenonT : public Material
{
public:
  XenonT(const InputParameters & parameters);

protected:
  /** 
   * Necessary override.  This is where the values of the properties
   * are computed.
   */
  virtual void computeQpProperties() override;
private:

    // The thermal conductivity 
  MaterialProperty<Real> & _thermal_conductivity; // for u-10 Mo

  // const VariableGradient & _dep_variable; // this will get the temperature 
  //std::vector<const VariableValue> & _dep_variable;
  //VariableValue<Real> & _dep_variable;
   const VariableValue & _dep_variable;
};

#endif //XENONT_H
\end{lstlisting}
\pagebreak
\texttt{HeatFlux.h}
\lstset{style=cpp}
\begin{lstlisting}{style=dflt}
ifndef HEATFLUX_H
#define HEATFLUX_H

#include "AuxKernel.h"

//Forward Declarations
class HeatFlux;

template<>
InputParameters validParams<HeatFlux>();

/**
 * Constant auxiliary value
 */
class HeatFlux : public AuxKernel
{
public:
  HeatFlux(const InputParameters & parameters);

  virtual ~HeatFlux() {}

protected:
  /** 
   * AuxKernels MUST override computeValue.  computeValue() is called on
   * every quadrature point.  For Nodal Auxiliary variables those quadrature
   * points coincide with the nodes.
   */
  virtual Real computeValue() override;


  /// The gradient of a coupled variable
  const VariableGradient & _temperature_gradient;

  /// Holds the permeability and viscosity from the material system
  const MaterialProperty<Real> & _t_thermal_conductivity;
  // const MaterialProperty<Real> & _viscosity;

    int _component;
};
#endif // end of HEATFLUX_H
\end{lstlisting}
\pagebreak
\lstset{style=cpp}
\texttt{TGradient.h}
\begin{lstlisting}{style=cpp}
ifndef TGRADIENT_H
#define TGRADIENT_H

#include "AuxKernel.h"

//Forward Declarations
class TGradient;

template<>
InputParameters validParams<TGradient>();

/**
 * Constant auxiliary value
 */
class TGradient : public AuxKernel
{
public:
  TGradient(const InputParameters & parameters);

  virtual ~TGradient() {}

protected:
  /** 
   * AuxKernels MUST override computeValue.  computeValue() is called on
   * every quadrature point.  For Nodal Auxiliary variables those quadrature
   * points coincide with the nodes.
   */
  virtual Real computeValue() override;


  /// The gradient of a coupled variable
  const VariableGradient & _temperature_gradient;

  /// Holds the permeability and viscosity from the material system
  //const MaterialProperty<Real> & _t_thermal_conductivity;
  int _component;
  // const MaterialProperty<Real> & _viscosity;
};

#endif // end of TGRADIENT_H
\end{lstlisting}
\pagebreak

\section{ATOMPAW}
This is an input file for Atompaw to generate the paw database for electronic structure calculation.
\lstset{style=atpw}
\begin{lstlisting}
U 92
GGA-PBE scalarrelativistic loggridv4 700 
7 6 6 5 0 
6 2 1   
5 3 3   
0 0 0 
c   !1s2    1   
c   !2s2    2   
c   !3s2    3   
c   !4s2    4   
c   !5s2    5   
v   !6s2    6       1
v   !7s2    7       2
c   !2p6    8   
c   !3p6    9   
c   !4p6    10  
c   !5p6    11    
v   !6p6    12      3   
c   !3d10   13  
c   !4d10   14  
c   !5d10   15  
v   !6d1    16      4   
c   !4f14   17  
v   !5f3    18      5   
3
2.5 2.02 1.5 1.8    !rpaw, rshape, rvloc, rcore, in a.u 
n
y               !one additional p partial wave
0.00
n               !no additional p partial wave
y               !one additional d partial wave
0.2             !
n               !no additional d partial wave
y               !one additional f partial wave
3.0
n               !no additional f partial wave
custom  rrkj besselshape    !UPF format needs bessels    
4 0.0 !bessel               !l quantum number, reference energy (Ry), bessel for simplicity
1.3     !1 s            ! rc s partial wave
1.3     !2 s            ! rc for s
2.5     !3 p            ! rc for p
2.5     !4 p            ! rc for p
1.55    !5 d            ! rc for d, 1.3 gives very good result
1.55    !6 d        ! rc for  d,    1.3 gives very good result, 4.0 energy
2.5     !7 f        ! rc for f
2.5     !8 f        ! rc for  f
PWSCFOUT
default
0
\end{lstlisting}

\section{DFT Calculation}
\subsection{SCF of a primitive cell}
This is sample input file to calculate the self consistent field study using Quantum Espresso. It contains the lattice parameter of \textalpha-uranium.
\lstset{style=deflt}
\begin{lstlisting}
&CONTROL
                       title = 'alphaU'
                 calculation = 'scf'
                restart_mode = 'from_scratch'
                      outdir = '/home/iasir/quantum_espresso/Uranium_PAW/rvloc_96/y_value_plot/lat01'
                  pseudo_dir = '/home/iasir/quantum_espresso/Uranium_PAW/rvloc_96/y_value_plot/lat01'
                      prefix = 'alphaU'
                   verbosity = 'default'
                     tstress = .true.
                     tprnfor = .true.
 /
 &SYSTEM
                       ibrav = 0 
                          A  = 2.8383
                         at  = 2 
                        ntyp = 1 
                        ntyp = 1 
                     ecutwfc = 50
                     ecutrho = 250 
                    smearing = 'methfessel-paxton'
 /
 &ELECTRONS
             diagonalization = 'david'
 /
 &IONS

 /
CELL_PARAMETERS {alat}
  0.500000000000000  -1.031658697792153   0.000000000000000 
  0.500000000000000   1.031658697792153   0.000000000000000 
  0.000000000000000   0.000000000000000   1.73379271551118
ATOMIC_SPECIES
    U  238.0290000000  U.GGA-PBE-paw.UPF

ATOMIC_POSITIONS {crystal}
U   0.09860000000000000  -0.0986000000000000  -0.250000000000000 
U  -0.09860000000000000    0.0986000000000000   0.250000000000000  

K_POINTS automatic
20  20  26   0 0 0 

\end{lstlisting}

\pagebreak
\subsection{Supercell of uranium}
The following input is for performing a relax calcuation using supercell of \textgamma-uranium with molybdenum and xenon.
\lstset{style=atpw}
\begin{lstlisting}
&CONTROL
                                title  = 'supercellgammaU'
                           calculation = 'relax'
                          restart_mode = 'from_scratch'
                         outdir = '/group/hammond/sysyphus/3x3/2prim-2nd/im02',
                            pseudo_dir = '/group/hammond/sysyphus/pp_dir',
                                prefix = 'supercell_gammaU'
                         etot_conv_thr = 1.0D-6
                         forc_conv_thr = 1.0D-6
                             verbosity = 'default'
                               tstress = .true.
                               tprnfor = .true.
                                nstep  = 200
/
&SYSTEM
                                 ibrav = 0
                                     A = 10.35
                                   nat = 53
                                  ntyp = 3
                               ecutwfc = 50
                               ecutrho = 260
                             occupations = 'smearing'
                                   degauss = 0.02
                                  smearing = 'mp'
/
&ELECTRONS
           diagonalization = 'david'
         electron_maxstep  = 900
               mixing_beta = 0.1
/
&IONS

/
&CELL

/
CELL_PARAMETERS {alat}
 1.00   0.00    0.00
 0.00   1.00    0.00
 0.00   0.00    1.00

ATOMIC_SPECIES
   U  238.02800   U.GGA-PBE-paw.UPF
   Xe 131.29      Xe.GGA-PBE-paw.UPF
   Mo 95.94       Mo.GGA-PBE-paw.UPF

ATOMIC_POSITIONS (crystal)
U        0.000000000   0.000000000   0.000000000    0   0   0
U        0.210984461   0.204688901   0.229195736
U        0.000000000   0.000000000   0.333333333    0   0   0
U        0.000000000   0.000000000   0.666666667    0   0   0
U        0.333333333   0.000000000   0.000000000    0   0   0
U        0.666666667   0.000000000   0.000000000    0   0   0
U        0.000000000   0.333333333   0.000000000    0   0   0
U        0.000000000   0.666666667   0.000000000    0   0   0
U        0.137231091   0.153757182   0.530873483
U        0.174376372   0.129124170   0.859923969
U        0.526508957   0.134953713   0.159596399
U        0.862089662   0.133706169   0.175813941
U        0.158683128   0.524614679   0.202129442
U        0.148170903   0.848856345   0.127554382
U        0.000000000   0.333333333   0.333333333    0   0   0
U        0.000000000   0.666666667   0.666666667    0   0   0
U        0.000000000   0.666666667   0.333333333    0   0   0
U        0.000000000   0.333333333   0.666666667    0   0   0
U        0.333333333   0.000000000   0.333333333    0   0   0
U        0.666666667   0.000000000   0.666666667    0   0   0
U        0.333333333   0.000000000   0.666666667    0   0   0
U        0.666666667   0.000000000   0.333333333    0   0   0
U        0.333333333   0.333333333   0.000000000    0   0   0
U        0.666666667   0.666666667   0.000000000    0   0   0
U        0.666666667   0.333333333   0.000000000    0   0   0
U        0.333333333   0.666666667   0.000000000    0   0   0
U        0.137231091   0.153757182   0.530873483
U        0.174376372   0.129124170   0.859923969
U        0.526508957   0.134953713   0.159596399
U        0.862089662   0.133706169   0.175813941
U        0.158683128   0.524614679   0.202129442
U        0.148170903   0.848856345   0.127554382
U        0.000000000   0.333333333   0.333333333    0   0   0
U        0.000000000   0.666666667   0.666666667    0   0   0
U        0.000000000   0.666666667   0.333333333    0   0   0
U        0.000000000   0.333333333   0.666666667    0   0   0
U        0.333333333   0.000000000   0.333333333    0   0   0
U        0.666666667   0.000000000   0.666666667    0   0   0
U        0.333333333   0.000000000   0.666666667    0   0   0
U        0.666666667   0.000000000   0.333333333    0   0   0
U        0.333333333   0.333333333   0.000000000    0   0   0
U        0.666666667   0.666666667   0.000000000    0   0   0
U        0.666666667   0.333333333   0.000000000    0   0   0
U        0.333333333   0.666666667   0.000000000    0   0   0
U        0.179014193   0.490220525   0.515819257
U        0.211529991   0.769018260   0.787411348
U        0.174791835   0.807115343   0.456517964
U        0.138750516   0.465098823   0.842421942
U        0.460277126   0.148072446   0.492479910
U        0.853185555   0.205057669   0.814927365
U        0.517574914   0.155283842   0.808674767
U        0.797006954   0.141631469   0.505322942
U        0.472193561   0.445111610   0.164972643
U        0.797427656   0.818792499   0.205498630
U        0.849105689   0.447077449   0.157067159
U        0.477316581   0.824076376   0.198019584
U        0.357811109   0.358247142   0.377740929
Mo       0.317120589   0.314631326   0.680448732
Mo       0.607263578   0.351682511   0.642824966
U        0.340594168   0.659454747   0.325699754
U        0.361075015   0.621507036   0.637398808
Mo       0.660724290   0.302342735   0.343336983
U        0.864625684   0.823864687   0.866239386
U        0.857300550   0.828682348   0.537389462
U        0.459239378   0.501675726   0.851462699
U        0.787353995   0.504077862   0.849782744
U        0.483651729   0.821500104   0.546062842
U        0.525967152   0.834454416   0.872311707
U        0.803327765   0.439115905   0.558594712
U        0.678734725   0.684452016   0.648081893
Xe       0.649495337   0.596675850   0.354354426



!Xe    0.500000000         0.500000000         0.500000000

K_POINTS automatic
4 4 4       0 0 0
\end{lstlisting}
\pagebreak
\subsection{Nudged Elastic Band Calculation}
The following input file is to perform nudged elastic band calculation using Quantum Espresso.
\lstset{style=atpw}
\begin{lstlisting}
BEGIN
BEGIN_PATH_INPUT
&PATH
  restart_mode      = 'from_scratch',
  string_method     = 'neb',
  nstep_path        = 100,
  ds                = 1.00,
  opt_scheme        = "broyden",
  num_of_images     = 5,
  k_max             = 0.6169D0,
  k_min             = 0.6169D0,
  CI_scheme         = "auto",
  path_thr          = 0.1D0,
/
END_PATH_INPUT
BEGIN_ENGINE_INPUT
&CONTROL
 outdir = '/group/hammond/sysyphus/3x3/2prim-2nd/neb',
 pseudo_dir = '/group/hammond/sysyphus/pp_dir',
 prefix = 'xe01neb' ,
 verbosity = 'high' ,
 etot_conv_thr = 1e-6 ,
 forc_conv_thr = 1e-5 ,
 nstep = 200 ,
 tstress = .true.,
 tprnfor = .true.,
 max_seconds = 1.0D+18,
/
&SYSTEM
                       ibrav = 0,
                           A = 10.35,
                         nat = 53, 
                        ntyp = 3,
                     ecutwfc = 50, 
                     ecutrho = 260,
                 occupations = 'smearing',
                     degauss = 0.02,
                    smearing = 'mp',
/
&ELECTRONS
            electron_maxstep = 900,
                    conv_thr = 1e-7 ,
                 mixing_beta = 0.1 ,
             diagonalization = 'david' ,
/
&IONS
/
ATOMIC_SPECIES
   U  238.02800   U.GGA-PBE-paw.UPF
   Xe 131.29      Xe.GGA-PBE-paw.UPF
   Mo 95.94       Mo.GGA-PBE-paw.UPF
BEGIN_POSITIONS
FIRST_IMAGE
ATOMIC_POSITIONS (crystal)
U        0.000000000   0.000000000   0.000000000    0   0   0
U        0.175109924   0.144562064   0.206801126
U        0.000000000   0.000000000   0.333333333    0   0   0
U        0.000000000   0.000000000   0.666666667    0   0   0
U        0.333333333   0.000000000   0.000000000    0   0   0
U        0.666666667   0.000000000   0.000000000    0   0   0
U        0.000000000   0.333333333   0.000000000    0   0   0
U        0.000000000   0.666666667   0.000000000    0   0   0
U        0.172966137   0.162988305   0.549703444
U        0.170611892   0.151612185   0.878052143
U        0.495257014   0.208922221   0.160293502
U        0.831518767   0.131820848   0.145403232
U        0.187663248   0.516383543   0.207043480
U        0.141733995   0.833501726   0.124057856
U        0.000000000   0.333333333   0.333333333    0   0   0
U        0.000000000   0.666666667   0.666666667    0   0   0
U        0.000000000   0.666666667   0.333333333    0   0   0
U        0.000000000   0.333333333   0.666666667    0   0   0
U        0.333333333   0.000000000   0.333333333    0   0   0
U        0.666666667   0.000000000   0.666666667    0   0   0
U        0.333333333   0.000000000   0.666666667    0   0   0
U        0.666666667   0.000000000   0.333333333    0   0   0
U        0.333333333   0.333333333   0.000000000    0   0   0
U        0.666666667   0.666666667   0.000000000    0   0   0
U        0.666666667   0.333333333   0.000000000    0   0   0
U        0.333333333   0.666666667   0.000000000    0   0   0
U        0.151218039   0.506822556   0.548576738
U        0.146390108   0.829542179   0.788362872
U        0.185174494   0.843342696   0.464715498
U        0.143670888   0.494112883   0.870258462
U        0.521448430   0.133334423   0.501328591
U        0.833500163   0.136787016   0.798434778
U        0.495145627   0.121151838   0.844025002
U        0.863545497   0.168786689   0.469156772
U        0.509274324   0.548329765   0.145881356
U        0.825440636   0.779984312   0.192980330
U        0.842423236   0.466015059   0.136756583
U        0.534458472   0.865141497   0.169291879
U        0.310602920   0.323532531   0.315012292
Mo       0.328606490   0.314883686   0.687255302
Mo       0.674009341   0.305267841   0.662448637
U        0.326631011   0.679870122   0.331479707
U        0.371644890   0.679274954   0.665208460
Mo       0.672193000   0.303810812   0.317212538
U        0.820847752   0.816237464   0.833790833
U        0.780071628   0.836705518   0.490628528
U        0.493041587   0.450284430   0.842990907
U        0.818302670   0.476570581   0.798002061
U        0.497363584   0.801441831   0.487806684
U        0.478662001   0.802952159   0.852807185
U        0.852755368   0.505332549   0.451217549
U        0.687902994   0.670557165   0.683653268
Xe       0.536400212   0.492947199   0.459667944
LAST_IMAGE
ATOMIC_POSITIONS (crystal)
U        0.000000000   0.000000000   0.000000000    0   0   0
U        0.211003121   0.204403180   0.229059646
U        0.000000000   0.000000000   0.333333333    0   0   0
U        0.000000000   0.000000000   0.666666667    0   0   0
U        0.333333333   0.000000000   0.000000000    0   0   0
U        0.666666667   0.000000000   0.000000000    0   0   0
U        0.000000000   0.333333333   0.000000000    0   0   0
U        0.000000000   0.666666667   0.000000000    0   0   0
U        0.137157542   0.154100490   0.530986252
U        0.174212597   0.129079723   0.860099413
U        0.526361077   0.134914785   0.159501081
U        0.862023067   0.133697598   0.175888456
U        0.158401680   0.524202200   0.202384340
U        0.148244066   0.848143594   0.127034040
U        0.000000000   0.333333333   0.333333333    0   0   0
U        0.000000000   0.666666667   0.666666667    0   0   0
U        0.000000000   0.666666667   0.333333333    0   0   0
U        0.000000000   0.333333333   0.666666667    0   0   0
U        0.333333333   0.000000000   0.333333333    0   0   0
U        0.666666667   0.000000000   0.666666667    0   0   0
U        0.333333333   0.000000000   0.666666667    0   0   0
U        0.666666667   0.000000000   0.333333333    0   0   0
U        0.333333333   0.333333333   0.000000000    0   0   0
U        0.666666667   0.666666667   0.000000000    0   0   0
U        0.666666667   0.333333333   0.000000000    0   0   0
U        0.333333333   0.666666667   0.000000000    0   0   0
U        0.178710377   0.490990744   0.516187076
U        0.211424168   0.769055779   0.787371675
U        0.175042054   0.807911917   0.455992431
U        0.138797282   0.465117394   0.842521779
U        0.460172734   0.148226288   0.492431541
U        0.853214362   0.204001135   0.813615314
U        0.517320360   0.155236389   0.808583066
U        0.796831649   0.141536092   0.504982369
U        0.472082596   0.445170387   0.165199942
U        0.797246556   0.819083198   0.205717103
U        0.848947010   0.446752941   0.157261389
U        0.477337922   0.823974130   0.198045419
U        0.357415253   0.358227015   0.377748970
Mo       0.316993580   0.314763082   0.680421155
Mo       0.607201462   0.351778562   0.642719282
U        0.340550068   0.659420403   0.325593019
U        0.361166032   0.621614060   0.637522343
Mo       0.660553429   0.302407475   0.343460369
U        0.864645732   0.824012823   0.866480839
U        0.857095779   0.828386743   0.537288219
U        0.459295873   0.501704484   0.851504670
U        0.787396737   0.503369090   0.850267568
U        0.483789070   0.821735869   0.546285283
U        0.525937071   0.834449910   0.872642503
U        0.803265518   0.439297450   0.559004083
U        0.678713059   0.684639768   0.648649342
Xe       0.649346200   0.596837093   0.354823657
END_POSITIONS
K_POINTS automatic
4 4 4  0 0 0
CELL_PARAMETERS {alat}
 1.00   0.00    0.00
 0.00   1.00    0.00
 0.00   0.00    1.00
END_ENGINE_INPUT
END

\end{lstlisting}


\newpage
\subsection{A sample code to run distorted lattice in quantum espresso}
The following code sample is valid for bcc lattice only.
\lstset{style=pythn}
\begin{lstlisting}
import os, subprocess
from glob import glob
import re
import matplotlib.pyplot as plt 
import numpy as np

def matprintf(x):
    for i in range(len(x)):
         print ('{:16.13f} {:16.13f} {:16.13f}'.format(x[i][0],x[i][1],x[i][2]))

#d = np.array([3.42, 3.43, 3.44, 3.455, 3.45, 3.465, 3.46, 3.475, 3.47, 3.48, 3.445, 3.4425])
d = np.array([0,0.01,-0.01,0.005,-0.005,0.0075,-0.0075,0.00625,-0.00625,0.0025,-0.0025,0.00375,-0.00375,0.00125,-0.00125,0.02,-0.02])
    
        # Run with different ecutwfc
for i in range(len(d)):
    dd = d[i]
    A12 = dd
    A21 = dd
    A33 = np.float128(1/((1-dd**2)))
     #crystal axis for bcc
    R = np.array([[-0.5,0.5,0.5],[0.5,-0.5,0.5],[0.5,0.5,-0.5]],dtype=np.float128)
    #distortion matrix for monoclinic distortion
    Dmono = np.array([[1,A12,0.0],[A21,1,0.0],[0.0,0.0,A33]],dtype=np.float128)
    Rprime = np.matmul(R,Dmono)


    filename = 'U_qepp_gamma.scf' 
    pseudo = '/home/scruffy/rafi/Desktop/rafi/quantum_espresso/supercell_study/pp_dir'
    with open(filename + str(d[i]) + '.in','w') as f:
        f.write("&control\n")
        f.write("title = 'U_gamma'\n")
        f.write("    calculation = 'relax'\n")
        f.write("    restart_mode = 'from_scratch'\n")
        f.write("    pseudo_dir = '{:s}' \n".format(pseudo))
        f.write("    outdir = '{:s}'\n".format(pseudo))
        f.write("    prefix = 'gammaU'\n")
        f.write("    etot_conv_thr = 1.0e-8\n")
        f.write("    forc_conv_thr = 1.0e-8 \n")
        #f.write("    tstress = .true.\n")
                #f.write("    tprnfor = .true.\n")
        f.write("/\n")
        f.write(" ")
        f.write("&system\n")
        f.write("    ibrav = 0\n")
        f.write("       A = 3.459\n")
        #f.write("        A = {:3.10f}\n" .format(d[i]))
        f.write("    nat = 1\n")
        f.write("    ntyp = 1\n")
        f.write("    ecutwfc = 50 \n")
        f.write("    ecutrho = 250 \n")
        f.write("    smearing = 'methfessel-paxton'\n")
                #f.write("    smearing = 'gauss'\n")
                #f.write("    degauss = 0.01\n")
        f.write("/\n")
        f.write(" ")
        f.write("&electrons\n")
        f.write("    diagonalization = 'david'\n")
                #f.write("    conv_thr = 1.0e-8\n")
                #f.write("    mixing_beta = 0.7\n")
        f.write("/\n")
        f.write(" ")
        f.write("&IONS\n")
        f.write("/\n")
        f.write(" ")
        f.write("&CELL\n")
        f.write("/\n")
        f.write("CELL_PARAMETERS {alat} \n")
        #f.write("  -0.5    0.5   0.5 \n")
        #f.write("   0.5   -0.5   0.5 \n")
        #f.write("   0.5    0.5  -0.5 \n")
    
        for j in range (len(Rprime)) :
                 f.write ('{:16.13f} {:16.13f} {:16.13f} \n'.format(Rprime[j][0],Rprime[j][1],Rprime[j][2]))
    
        f.write(" \n")
        f.write("ATOMIC_SPECIES\n")
        f.write("U 238.029  U.GGA-PBE-paw.UPF\n")
        f.write(" \n")
        f.write("ATOMIC_POSITIONS {crystal}\n")
        f.write("U  0.000  0.000 0.000\n")
        f.write(" \n")
        f.write("K_POINTS automatic\n")
        f.write("35 35 35 0 0 0\n")
    command = ('mpiexec -n 8 pw.x -i ' + filename + str(d[i]) + '.in > ' + filename + str(d[i]) + '.out' )
    print(command)
    p = subprocess.Popen(command, shell=True)
    p.wait()

\end{lstlisting}


