\documentclass{MUthesis}
\usepackage{amsmath}
%\usepackage{pxfonts}
\usepackage{newpxtext,newpxmath}
\usepackage[toc,page]{appendix}
%\usepackage{textgreek}
%\usepackage{appendix}

\newcommand*{\textalpha}{\text{\ensuremath{\alphaup}}}

\title{Hi!}
\author{My Butt}

\begin{document}
%\maketitle
\tableofcontents

\chapter{Preface}
blah blah
blah blah
blah blah
blah blah
blah blah
blah blah
blah blah
blah blah
blah blah
blah blah
blah blah
\mainmatter

\chapter{Main Chapter}
\section{A section}
\subsection{blah}
\subsubsection{blah}
\paragraph{blah}
\subparagraph{blah}
\section{Another section}
\chapter{Another Main Chapter}
\section{A section}
\section{Another section}
\section{Yet Another section}

\chapter{Chapter with sections}
\section{A section}
\section[Another section with Greek $\alpha$ in it]
        {Another section with Greek \boldmath$\alpha$ in it}
\section{Yet Another section}
%\begin{subappendices}
\clearpage
\addcontentsline{toc}{section}{Appendix: Calculation of Elastic Parameters of $\alphaup$-uranium}
\section*{\boldmath Appendix: Calculation of Elastic Parameters of $\alphaup$-uranium}\label{appen_elalpha}
In this appendix, we will discuss the stress-strain relationships that was used to calculated the nine independent elastic constants of \textalpha-uranium. The base-centered orthorhombic phase of uranium has three lattice parameters $a$, $b$, and $c$, with the Bravais lattice matrix.

\begin{equation}\label{eq_lattic_alphaU}
\mathbf{R} = \begin{pmatrix}
			\frac{a}{2} & -\frac{b}{2} & 0 \\
			\frac{a}{2} & \frac{b}{2} & 0 \\
			0			&    0        & c 
			\end{pmatrix}
\end{equation}

%\end{subappendices}

\chapter{Extra Chapter}

\appendix

\chapter{Hello!}

\end{document}
